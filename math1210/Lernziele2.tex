\documentclass{article}

\setlength{\hoffset}{-2cm}
\setlength{\voffset}{-1.5cm}
\addtolength{\textwidth}{4cm}
\addtolength{\textheight}{3cm}
%\setlength{\parindent}{0cm}

\usepackage[latin1]{inputenc}
\usepackage{amsmath, ulem, setspace, fancyhdr, nicefrac}
\usepackage{nicefrac}
\usepackage[T1]{fontenc}
\usepackage[latin1]{inputenc}
\usepackage{textcomp}
\usepackage{lastpage}
\usepackage{amssymb,amsfonts}
\usepackage{latexsym}
\usepackage{amsmath}
\usepackage{amsthm}
\usepackage[all]{xy}
\usepackage{mathrsfs}
\usepackage{stmaryrd}
\usepackage{oldgerm}
\usepackage{euscript}
\usepackage{eufrak}
\usepackage{graphics}
\usepackage{wrapfig}
\usepackage{fancybox}
\usepackage{graphicx}
\input xy
\xyoption{all}

\theoremstyle{plain}
\newtheorem{theo}{Theorem}
\newtheorem{lem}[theo]{Lemma}
\newtheorem{cor}[theo]{Corollary}
\newtheorem{prop}[theo]{Proposition}
\newtheorem{theodefi}[theo]{Theorem and Definition}
\newtheorem{conj}[theo]{Conjecture}

\theoremstyle{remark}
\newtheorem{rem}[theo]{Remark}

\theoremstyle{definition}
\newtheorem{defi}[theo]{Definition}
\newtheorem{exs}[theo]{Examples}
\newtheorem{ex}[theo]{Example}
\newtheorem{nota}[theo]{Notation}
\newtheorem{para}[theo]{}

\newenvironment{pr}{\textsc{Proof}}{}


\pagestyle{fancy}
\lhead{Math1210}
\chead{Study guide 2}
\rhead{18th September 2013}
\lfoot{Ertl Veronika}
\cfoot{\textit{University of Utah}}
\rfoot{\textit{Mathematics Department}}

\input{MathOperators}

\begin{document}

\subsection*{Differentiation}

The first derivative at a point calculates the slope of a curve at that point. The slope oif th curve at a point is the slope of the tangent line. We can find it using limits.

\begin{defi} Let $f$ be a function and $x_0$ an element of the domain. If the limit
$$\lim_{x\rightarrow x_0}\frac{f(x)-f(x_0)}{x-x_0}$$
exists, we call it first derivative of $f$ at the point $x_0$ and denote it by $f'(x_0)$\end{defi}

Sometimes the derivative is denoted as 
$$\frac{df}{dx}(x)\quad\text{ or }\quad D_xf(x)$$
We say that $f$ is differentiable at $x_0$, if the above limit exists. If it is differentiable at all point of the domain, we say $f$ is differentiable. In that case, we are looking for a function $f':x\mapsto m(x)$ where $m(x)$ is the slope of $f$ at $x$. 

\begin{theo} If a function is differentiable at a point, it is also continuous at this point.\end{theo}
But the converse is not true! If a function is discontinuous, it is not differentiable at this point. It is alos not differentible at a cusp or a pole. 

\subsection*{Differentiation rules}

\begin{enumerate} \item Constant function rule: If $f(x)=k$ for $k\in \RR$ then
$$f'(x)= 0$$
\item Identity function rule: If $f(x)=x$ then
$$f'(x)=1$$
\item Power rule: For $f(x)=x^n$ with $n\in\NN$
$$f'(x)=nx^{n-1}$$
\item Constant multiple rule: If $k\in\RR$ and $f(x)$ is $f'(x)$ exists then
$$\left(kf(x)\right)'=kf'(x)$$
\item Sum and difference rule: If $f$ and $g$ are differentiable then
$$(f+g)'(x) =f'(x)+g'(x)$$
\item Square root rule: If $f(x)=\sqrt{x}$ then
$$f'(x)=\frac{1}{2\sqrt{x}}$$
\item Product rule: If $f$ and $g$ are differentiable then
$$\left(f\cdot g\right)'(x)=f'(x)g(x)+g'(x)f(x)$$
\item Quotient rule: If $f$ and $g$ are differentiable and $g(x)\neq 0$ then
$$\left(\frac{f}{g}\right)'(x)=\frac{g(x)f'(x)-f(x)g'(x)}{g^2(x)}$$
\item Trigonometric functions:
\begin{eqnarray*}\sin'(x)&=&\cos(x)\\
\cos'(x) &=& -\sin(x)\\
\tan'(x) &=& \sec^2(x)\\
\sec'(x) &=& \sec(x)\tan(x)\\
\cot'(x) &=& -\csc^2(x)\\
\csc'(x) &=& -\csc(x)\cot(x)\end{eqnarray*}
\item Chain rule: Let $f$ and $g$ be differentiable. Then
$$(f\circ g)'(x)=f'(g(x))\cdot g'(x)$$
\end{enumerate}

\subsection*{Higher derivatives}

To get the second derivative, you need the first derivative first. THe process is recursive.

\subsection*{Implicit differentiation}

Given an equation $f(y)=g(x)$. This can be seen as a function $y(x)$ given implicitely. Oftentimes we cannot solve directly for $y$ in terms of $x$. But we can try to differentiate it.
\begin{eqnarray*}\frac{d}{dx}\left(f(y)\right) &=& \frac{d}{dx}g(x)\\
f'(y(x))y'(x) &=& g'(x)\\
y'(x) &=& \frac{g'(x)}{f'(y)}\end{eqnarray*}

\subsection*{Related Rates}






\end{document}A
