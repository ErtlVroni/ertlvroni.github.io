\documentclass{article}

\setlength{\hoffset}{-2cm}
\setlength{\voffset}{-1.5cm}
\addtolength{\textwidth}{4cm}
\addtolength{\textheight}{3cm}
%\setlength{\parindent}{0cm}

\usepackage[latin1]{inputenc}
\usepackage{amsmath, ulem, setspace, fancyhdr, nicefrac}
\usepackage{nicefrac}
\usepackage[T1]{fontenc}
\usepackage[latin1]{inputenc}
\usepackage{textcomp}
\usepackage{lastpage}
\usepackage{amssymb,amsfonts}
\usepackage{latexsym}
\usepackage{amsmath}
\usepackage{amsthm}
\usepackage[all]{xy}
\usepackage{mathrsfs}
\usepackage{stmaryrd}
\usepackage{oldgerm}
\usepackage{euscript}
\usepackage{eufrak}
\usepackage{graphics}
\usepackage{wrapfig}
\usepackage{fancybox}
\usepackage{graphicx}
\input xy
\xyoption{all}

\theoremstyle{plain}
\newtheorem{theo}{Theorem}%[section]
\newtheorem{lem}[theo]{Lemma}
\newtheorem{cor}[theo]{Corollary}
\newtheorem{prop}[theo]{Proposition}
\newtheorem{theodefi}[theo]{Theorem and Definition}
\newtheorem{conj}[theo]{Conjecture}

\theoremstyle{remark}
\newtheorem{rem}[theo]{Remark}

\theoremstyle{definition}
\newtheorem{defi}[theo]{Definition}
\newtheorem{exs}[theo]{Examples}
\newtheorem{ex}[theo]{Example}
\newtheorem{nota}[theo]{Notation}
\newtheorem{para}[theo]{}

\newenvironment{pr}{\textsc{Proof}}{}


\pagestyle{fancy}
\lhead{Math1210}
\chead{Study guide 1}
\rhead{18th September 2013}
\lfoot{Ertl Veronika}
\cfoot{\textit{University of Utah}}
\rfoot{\textit{Mathematics Department}}

\input{MathOperators}

\begin{document}
\subsection*{Functions}
\begin{defi} A rule that assignes to each element in the domain exactly one element in the target is called a function.\end{defi}
One writes 
\begin{eqnarray*} f: \mathbb{D} &\rightarrow & \mathbb{T}\\ x &\mapsto& y=f(x)\end{eqnarray*}
The expression $y=f(x)$ is called equation of function. 
\begin{defi} A function of the form $f:x\mapsto y=mx+t$ with $m,t\in\RR$ fixed is said to be linear. $m$ and $t$ are called parameter of the function.\end{defi}
The parameter $t$ gives the distance to the abscisse ($x$-axis) and the $y$-intercept is given by $T=(0,t)$. The parameter $m$ gives the slope of the line and we have $\tan\rho=m$, where $\rho$ is the intersection angle. 

The zeros of the term $f(x)=0$ give the $x$-intercepts. 

Two lines $y=f(x)$ and $y=g(x)$ have either no intersection point (if they have the same slope) or one (if they have different slopes. In the latter case one calculates the point of intersection by setting
$$f(x)=g(x),$$
solving this equation for $x$ and then plugging into $f$ or $g$ to solve for $y$. 

A function is invertible if it is injective, meaning that for $x_1\neq x_2$ we have necessarily $f(x_1)=\neq f(x_2)$. To find the inverese of a function, solve $y=f(x)$ for $x$ and interchange $x$ and $y$. One obtains the graph of the inverse function by taking the mirror image of the graph of the original function on the diagonal of the first quadrant. 

One can connect functions in in the following way:
\begin{enumerate} \item Addition: $f(x)+g(x)= (f+g)(x)$
\item Subtraction: $f(x)-g(x)=(f-g)(x)$
\item Multiplication: $f(x)\cdot g(x)= f\cdot g(x)$
\item Division: $\frac{f(x)}{g(x)}=\frac{f}{g}(x)$ for $g(x)\neq 0$. 
\item $f\circ g(x)=f(g(x))$ In general $f\circ g\neq g\circ f$.
\end{enumerate}

\subsection*{Limits}

\begin{defi} The limit of $f(x)$ as $x$ goes towards $x_0$
$$\lim_{x\rightarrow x_0}f(x)=f(x_0)$$
\end{defi}
If $f$ is not defined at the place $x_0$ the limit is the continuation of $f$ at this point. A limit exists exactly if the function can be continued at this point. One says that $f$ converges to $\widetilde{f}(x_0)$ as $x$ goes to $x_0$. 
There are right sided and left sided limits. If they coincide, the function can be continued at this point. 

\subsubsection*{Limit Theorems}

Assume $n\in\NN$, $k\in\RR$ and $f(x)$ and $g(x)$ have a limit as $x\rightarrow c$.

\begin{enumerate} \item $\lim_{x\rightarrow c}k=k$
\item $\lim_{x\rightarrow c}x=c$
\item $\lim_{x\rightarrow c}kf(x)=k\lim_{x\rightarrow c}f(x)$
\item $\lim_{x\rightarrow c}[f(x)\pm g(x)]=\lim_{x\rightarrow c}f(x)\pm\lim_{x\rightarrow c}g(x)$
\item $\lim_{x\rightarrow c}[f(x)g(x)]=\lim_{x\rightarrow c}f(x)\cdot\lim_{x\rightarrow c}g(x)$
\item $\lim_{x\rightarrow c}\frac{f(x)}{g(x)}=\frac{\lim_{x\rightarrow c}f(x)}{\lim_{x\rightarrow c}g(x)}$ provided $\lim_{x\rightarrow c}g(x)\neq 0$.
\item $\lim_{x\rightarrow c}f(x)^n = [\lim_{x\rightarrow c}f(x)]^n$
\item $\lim_{x\rightarrow c}\sqrt[n]{f(x)}=\sqrt[n]{\lim_{x\rightarrow c}f(x)}$
\item Substitution Theorem: If $f(x)$ is a polynomial or rational function, then $\lim_{x\rightarrow c}f(x)=f(c)$ assuming $f(c)$ is defined.
\item Squeeze Theorem: Let $f,g,h$ be functions satisfying $f(x)\leqslant g(x)\leqslant h(x)$ for all $x$ near $c$, except possibly at $x=c$. If $\lim_{x\rightarrow c}f(x)=\lim_{x\rightarrow c}h(x)=L$ then $\lim_{x\rightarrow c}g(x)=L$.
\end{enumerate}



\subsubsection*{Trigonometric limits.}

\subsubsection*{Limits at infinity, infinite limits}

\begin{defi} Let $f$ be defined on $[c,\infty)$ resp. $(-\infty,c]$ for some $c\in\RR$. We say that 
$$\lim_{x\rightarrow\infty\text{ or }-\infty} f(x) = L$$
if $\forall\epsilon>0\exists$ a corresponding number $M$ such that 
$$x>M\quad\text{ (or }x<M) \Rightarrow |f(x)-L|<\epsilon.$$
\end{defi}
\begin{defi} We say that 
$$\lim_{x\rightarrow c^+}f(x)=\infty$$
if $\forall$ positive numbers $M\exists$ a corresponding $\delta>0$ such that
$$0<x-c<\delta\Rightarrow f(x)>M.$$
\end{defi}
Similar for leftsided limits as well as $-\infty$.

\subsection*{Continuity}

\begin{defi} A function is contiuous at the point $x_0$ if for every $\epsilon>0$ there is $\delta>0$ such that for $|x-x_0|<\delta$ we hvae $|f(x)-f(x_0)|<\epsilon$.\end{defi}
A function is continuous on the domain, if it is continuous at each point of the domain. One can check continuity with the limit property.
$$\lim_{x\rightarrow x_0^+}f(x)=\lim_{x\rightarrow x_0^-}f(x)=\widetilde{f}(x_0)$$

Continuous functions:
\begin{enumerate}\item All polynomials are continuous everywhere.
\item All rational functions are continuous where they are defined.
\item Absolute value function is continuous.
\item $f(x)=\sqrt[n]{x}$ is continuous
\item Trigonometric functions are continuous.
\item If $f$ and $g$ are continuous at $c$ then so are
$$kf,\;f+g,\;f-g,\;fg,\;\frac{f}{g}\quad(\text{if }g(c)\neq 0),\;f^n,\;\sqrt[n]{f}.$$
\item If $g$ is contiuous at $c$ and $f$ is continuous at $g(c)$ then $f\circ g$ is continuous at $c$.
\end{enumerate}

Intermediate Value Theorem
\begin{theo} Let $f$ be a function defined on $[a,b]$ and $w\in[f(a),f(b)]$. If $f$ is continuous on $[a,b]$ then there exists a number $c\in[a,b]$ such that $f(c)=w$.\end{theo}




\end{document}A
